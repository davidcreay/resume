\documentclass[8pt,a4paper,ragged2e,withhyper]{altacv}
\geometry{left=1.5cm,right=1.5cm,top=1.5cm,bottom=1.5cm,columnsep=1.2cm}

\begin{document}
\noindent\name{David Reay}
\tagline{DevOps Enginner | Strategy \& Innovation}
%% You can add multiple photos on the left or right
%\photoR{2.8cm}{Globe_High}
% \photoL{2.5cm}{Yacht_High,Suitcase_High}

\personalinfo{%
  % Not all of these are required!
  \email{david@awsbot.com}
  \linkedin{davidreay}
%  \phone{000-00-0000}
%  \mailaddress{Åddrésş, Street, 00000 Cóuntry}
  \location{London, United Kingdom}
%  \homepage{www.homepage.com}
%  \twitter{@twitterhandle}

%  \github{your_id}
%  \orcid{0000-0000-0000-0000}
  %% You can add your own arbitrary detail with
  %% \printinfo{symbol}{detail}[optional hyperlink prefix]
  % \printinfo{\faPaw}{Hey ho!}[https://example.com/]

  %% Or you can declare your own field with
  %% \NewInfoFiled{fieldname}{symbol}[optional hyperlink prefix] and use it:
  % \NewInfoField{gitlab}{\faGitlab}[https://gitlab.com/]
  % \gitlab{your_id}
  %%
  %% For services and platforms like Mastodon where there isn't a
  %% straightforward relation between the user ID/nickname and the hyperlink,
  %% you can use \printinfo directly e.g.
  % \printinfo{\faMastodon}{@username@instace}[https://instance.url/@username]
  %% But if you absolutely want to create new dedicated info fields for
  %% such platforms, then use \NewInfoField* with a star:
  % \NewInfoField*{mastodon}{\faMastodon}
  %% then you can use \mastodon, with TWO arguments where the 2nd argument is
  %% the full hyperlink.
  % \mastodon{@username@instance}{https://instance.url/@username}
}

\makecvheader
%% Depending on your tastes, you may want to make fonts of itemize environments slightly smaller
% \AtBeginEnvironment{itemize}{\small}

%% Set the left/right column width ratio to 6:4.
\columnratio{0.6}

% Start a 2-column paracol. Both the left and right columns will automatically
% break across pages if things get too long.
\begin{paracol}{2}

\cvsection{Summary}
With over 5 years of experience in product management, I have a proven track record of driving product strategy and innovation. My expertise in leading cross-functional teams to deliver impactful products aligns with my passion for creating solutions that meet user needs and business goals.


\cvsection{Experience}
\cvevent{Senior Product Manager}{Plaid}{01/20/2021 -- Preent}{San Franscisco, California}
\begin{itemize}
\item Led the development and launch of a new financial data integration platform, increasing market penetration by 25\% within the first year.
\item Managed a cross-functional team of 10, including engineers and designers, to deliver the project on time and under budget.
\item Implemented a customer feedback loop that resulted in a 40\% improvement in customer satisfaction scores.
\item Conducted competitive analysis to inform product positioning, leading to a 15\% increase in product adoption.
\item Negotiated partnerships with key financial institutions, expanding the platform's reach and functionality.
\item Developed and executed a go-to-market strategy that resulted in the product being featured in major tech publications.
\end{itemize}

\divider

\cvevent{Product Manager}{Stripe}{06/2018 -- 12/2020}{San Franscisco, California}
\begin{itemize}
\item Oversaw the development of a new payment processing feature, resulting in a 20% increase in transaction volume.
\item Collaborated with the engineering team to streamline the development process, reducing time to market by 30%.
\item Led user research initiatives that informed product improvements, enhancing user experience and satisfaction.
\item Managed product backlog and prioritization, ensuring alignment with strategic business goals.
\item Facilitated cross-departmental communication to align product development with marketing and sales strategies.
\end{itemize}

\divider

\cvevent{Associate Product Manager}{Salesforce}{03/2026 -- 05/2018}{San Franscisco, California}
\begin{itemize}
\item Contributed to the development of a CRM feature that increased user engagement by 30%.
\item Assisted in conducting market research and analysis to guide product development strategies.
\item Supported senior product managers in roadmap planning and execution.
\item Participated in agile development processes, contributing to sprint planning and review meetings.
\end{itemize}


\cvsection{Education}
\cvevent{Master of Business Aministration (MBA)}{The Open University}{09/2021 -- 10/2025}{Online}
\divider

\cvevent{Bachelor of Science in Physics}{The Open University}{09/2006 -- 10/2011}{Online}


\cvsection{Languages}
\begin{minipage}[t]{0.5\linewidth}
  \cvlanguage{English}{Native}{5}
\end{minipage}
\hfill
\begin{minipage}[t]{0.5\linewidth}
  \cvlanguage{French}{Fluent}{4}
\end{minipage}

%% Switch to the right column. This will now automatically move to the second
%% page if the content is too long.
\switchcolumn

\cvsection{Key Achievements}
\cvachievement{\faRocket}{Launched Financial Data Platform}{Led the successful launch of a financial data integration platform at Plaid, significantly increasing market penetration.}
\divider
\cvachievement{\faStarHalf*}{Improved Customer Satisfaction}{Implemented a customer feedback loop at Plaid, improving customer satisfaction scores by 40\%.}
\divider
\cvachievement{\faMagic
}{Increased Transaction Volume}{Developed a new payment processing feature at Stripe, leading to a 20\% increase in transaction volume.}
\divider
\cvachievement{\faTrophy}{Streamlined Development Process}{Collaborated with engineering at Stripe to reduce time to market by 30\%, enhancing product competitiveness.}

\cvsection{Skills}
\cvtag{Hard-working}
\cvtag{Eye for detail}\\
\cvtag{Motivator \& Leader}
\cvtag{C++}
\cvtag{Embedded Systems}\\
\cvtag{Statistical Analysis}

\cvsection{Certification}
\cvcert{
Certified ScrumMaster (CSM)
}{
Gained expertise in Scrum methodologies and agile project management from Scrum Alliance.
}
\divider
\cvcert{
User Experience Design
}{
Completed a comprehensive course on UX design principles and practices from Coursera.
}


\cvsection{Interests}
\cvachievement{\faCheck}{
Financial Technology Innovation
}{
Passionate about leveraging technology to transform the financial services industry, driving efficiency and accessibility.
}


\end{paracol}

\end{document}
